\documentclass[12pt]{article}

\newcommand{\unit}[1]{\mathrm{#1}}
\newcommand{\m}{\unit{m}}
\newcommand{\s}{\unit{s}}
\newcommand{\mps}{\m\,\s^{-1}}
\newcommand{\mas}{\unit{mas}}
\newcommand{\yr}{\unit{yr}}
\newcommand{\maspyr}{\mas\,\yr^{-1}}

\begin{document}

\section*{Measuring astrometric quantities with a radial-velocity survey}

\noindent
\textbf{Abstract:}
With stellar radial-velocity measurement projects pushing towards
$0.1$ or even $0.01\,\mps$ precisions, the barycentric
correction---the contribution to radial-velocity signals from the
motion of the observatory with respect to the Solar System
barycenter---must be known extremely precisely.
At the same time, the brightest stars, which comprise some of the best
targets for radial-velocity planet searches, often do not have their
celestial positions and proper motions measured extremely precisely.
These two problems intersect, because the barycentric radial-velocity
correction depends sensitively on the angular position and velocity of
the target star.
Here we turn this challenge into an opportunity by showing that precise
radial-velocity measurements can be used, in concert with an accurate
Solar System ephemeris and geodesy, to infer positions and proper motions
for stars at the $100\,\mas$ and $10\,\maspyr$ levels, given a decade of
extremely good radial-velocity measurements.
We also show that precise knowledge of the stellar position and proper
motion is not required for radial-velocity planet search
and characterization.
If the target-star astrometric quantities are unknown or uncertain,
they must be fit simultaneously with the exoplanet orbit parameters.

\end{document}
